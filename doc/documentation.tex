%%%%%%%%%%%%%%%%%%%%%%%%%%%%%%%%%%%%%%%%%%%%%%%%%%%%%%%%%%
%
% DODATEČNÉ BALÍČKY PRO SAZBU
% Jejich užívání či neužívání záleží na libovůli autora 
% práce
%
%%%%%%%%%%%%%%%%%%%%%%%%%%%%%%%%%%%%%%%%%%%%%%%%%%%%%%%%%%

\documentclass[12pt]{report}


% Zařadit literaturu do obsahu
\usepackage[nottoc,notlot,notlof]{tocbibind}

% Umožňuje vkládání obrázků
\usepackage[pdftex]{graphicx}

% Odkazy v PDF jsou aktivní; navíc se automaticky vkládá
% balíček 'url', který umožňuje např. dělení slov
% uvnitř URL
\usepackage[pdftex]{hyperref}
\hypersetup{colorlinks=true,
  unicode=true,
  linkcolor=black,
  citecolor=black,
  urlcolor=black,
  bookmarksopen=true}

% Při používání citačního stylu csplainnatkiv
% (odvozen z csplainnat, http://repo.or.cz/w/csplainnat.git)
% lze snadno modifikovat vzhled citací v textu
\usepackage[numbers,sort&compress]{natbib}


% me package
\usepackage{float}
\usepackage{subfig}
\usepackage{siunitx}
\usepackage[czech]{babel}

%%%%%%%%%%%%%%%%%%%%%%%%%%%%%%%%%%%%%%%%%%%%%%%%%%%%%%%%%%
%
% VLASTNÍ TEXT PRÁCE
%
%%%%%%%%%%%%%%%%%%%%%%%%%%%%%%%%%%%%%%%%%%%%%%%%%%%%%%%%%%

\makeatletter
\def\@makechapterhead#1{%
  {\parindent \z@ \raggedright \normalfont
    \Huge\bfseries
    \ifnum \c@secnumdepth >\m@ne
        \hangindent=1.5em
        \noindent\hbox to 1.5em{\thechapter\hfil}%
    \fi%
    #1\par\nobreak
    \vskip 40\p@
  }}
\def\@makeschapterhead#1{%
  {\parindent \z@ \raggedright \normalfont
    \interlinepenalty\@M
    \Huge \bfseries  #1\par\nobreak
    \vskip 40\p@
  }}
\makeatother


\begin{document}
%

\begin{titlepage}
\centerline{\includegraphics[width=10cm]{img/logo.jpg}}
\begin{center}
\vspace{30px}
{\Huge
\textbf{Úvod do počítačových sítí}\\
\vspace{1cm}
}
{\Large
\textbf{KIV/UPS}\\
\vspace{1cm}
}
\vspace{1cm}
{\large
Pavel Třeštík\\
}
{\normalsize
A17B0380P
}
\end{center}
\vspace{\fill}
\hfill
\begin{minipage}[t]{7cm}
\flushright
\today
\end{minipage}
\end{titlepage}


%\maketitle
\tableofcontents



\chapter{Popis hry Dáma}
Jako téma semestrální práce jsem si zvolil hru Dáma. Zvolil jsem "klasickou" verzi. Tedy
hra se hraje na šachovnici s 64mi poli a vyhrává ten hráč, který první "sežere" všechny
kameny protivníka. Hráči mají kameny na černých polích šachovnice a smí se pohybovat pouze
diagonálně, tedy pouze po černých polích. Kamenů jsou dva druhy: pěšák a král.

\section{Pohyb}
Pěšák smí pouze o jedno políčko v před. Král smí o jedno políčko všemi směry. Tah hráče
končí po posunutí kamene.

TODO obrazek tahu

\section{Braní kamenů protivníka}
Pokud v cestě ku předu stojí kámen protivníka a je za ním prázdné políčko, tak jak král, tak pěšák 
přeskočí kámen protivníka na políčko za něj a kámen protivníka odstraní. Stále platí, že
pěšák smí brát pouze ku před, zatímco král může brát všemi směry.

TODO obrazek zrani

\section{Pohyb}
Pokud po tom, co hráč vzal protivníkův kámen, je v cestě další protivníkův kámen, za kterým
je volné místé, hráč táhne znovu a vezme tím tak 2 protívníkovy kameny v jednom tahu. Toto
je možné opakovat pokud je splněna ta podmínka, že v blízkosti kamene je protivníkův kámen
a za ním je volné políčko.

TODO: obrazek retezoveho brani
%
%
%
\chapter{Popis protokolu}
Protokol je posílán jako nešifrovaný text. Jednotlivé části jsou oddělené znakem '$|$'.
	\section{Obecný tvar zpráv}
Obecný tvar zprávy je:
\begin{itemize}
	\item ID\_hráče$|$INSTRUKCE$|$parameter$|$parametr...
\end{itemize}

\noindent Obecný tvar odpovědi serveru je: 

\begin{itemize}
	\item ID\_hráče$|$VYSLEDEK$|$kód\_výsledku$|$zpráva$|$parameter$|$parametr...
\end{itemize}


\noindent VYSLEDEK nabývá hodnok "OK" a "ERROR", lze ho považovat za typ instrukce.

\section{Popis instrukcí a odpovědí}
%
%
%
\subsection{CONNECT}
\subsubsection{Popis}
Připojí hráče k serveru. Pokud první zpráva od nového připojení není CONNECT,
tak server připojení odpojí. Při zadání ID hráče slouží k znovu připojení k serveru.\\

\subsubsection{Přesné formáty}
\begin{itemize}
	\item 0$|$CONNECT$|$username - hráč se připojuje poprvé
		\begin{itemize}
			\item username - jméno hráče
		\end{itemize}
	\item ID\_hráče$|$CONNECT$|$username - hráč se pokouší znovupřipojit
		\begin{itemize}
			\item username - jméno hráče
		\end{itemize}
\end{itemize}

\subsubsection{Kódy a zprávy odpovědí}
\begin{itemize}
	\item Pozitivní
		\begin{itemize}
			\item 201 - Connection success
			\item 202 - Reconnection success
		\end{itemize}
	\item Negativní
		\begin{itemize}
			\item 402 -
			\item 402 -
		\end{itemize}
\end{itemize}
%
%
%
\subsection{CREATE\_LOBBY}
\subsubsection{Popis}
Vytvoří nové lobby a přidá hráče, který ho vytvořil jako hráč 1. Hráč 1 má vždy
bílé kámeny.

\subsubsection{Přesné formáty}
\begin{itemize}
	\item ID\_$|$CREATE\_LOBBY$|$lobby\_name
		\begin{itemize}
			\item lobby\_name - jméno vytvářené místnosti
		\end{itemize}
\end{itemize}

\subsubsection{Kódy a zprávy odpovědí}
\begin{itemize}
	\item Pozitivní
		\begin{itemize}
			\item 201 - Successfully created lobby
		\end{itemize}
	\item Negativní
		\begin{itemize}
			\item 402 - Server failed to create lobby
			\item 403 - Lobby name is too long
		\end{itemize}
\end{itemize}
%
%
%
\subsection{JOIN\_GAME}
\subsubsection{Popis}
Připojí hráče do existující místnosti. Připojené hráče nastaví jako hráč 2, který má
černé kameny. Převede stav hry na stav probíhající.

\subsubsection{Přesné formáty}
\begin{itemize}
	\item ID\_hráče$|$JOIN\_GAME$|$lobby\_name
		\begin{itemize}
			\item lobby\_name - jméno místnosti pro připojení
		\end{itemize}
\end{itemize}

\subsubsection{Kódy a zprávy odpovědí}
\begin{itemize}
	\item Pozitivní
		\begin{itemize}
			\item 201 - Successfully joined game
		\end{itemize}
	\item Negativní
		\begin{itemize}
			\item 402 -
			\item 402 -
		\end{itemize}
\end{itemize}
%
%
%
\subsection{DELETE\_LOBBY}
\subsubsection{Popis}
Zruší vytvořené lobby, pokud v něm je pouze zakladatel lobby.

\subsubsection{Přesné formáty}
\begin{itemize}
	\item ID\_hráče$|$DELETE\_LOBBY
\end{itemize}

\subsubsection{Kódy a zprávy odpovědí}
\begin{itemize}
	\item Pozitivní
		\begin{itemize}
			\item 201 - Lobby deleted
		\end{itemize}
	\item Negativní
		\begin{itemize}
			\item 402 -
			\item 402 -
		\end{itemize}
\end{itemize}
%
%
%
\subsection{LOBBY}
\subsubsection{Popis}
Požadavek uživatele na získání lobby, dostupných k připojení.

\subsubsection{Přesné formáty}
\begin{itemize}
	\item ID\_hráče$|$LOBBY
\end{itemize}

\subsubsection{Kódy a zprávy odpovědí}
\begin{itemize}
	\item Pozitivní
		\begin{itemize}
			\item 201 - Available lobbies
		\end{itemize}
	\item Negativní
		\begin{itemize}
			\item 402 -
			\item 402 -
		\end{itemize}
\end{itemize}
%
%
%
\subsection{TURN}
\subsubsection{Popis}
Provede tah nebo sekvenci tahů. Kontaktuje protivníka o hráčovo tazích.

\subsubsection{Přesný formáty}
\begin{itemize}
	\item ID\_hráče$|$TURN$|$paramter\_1$|$parametr\_2$|$...$|$parameter\_30
		\begin{itemize}
			\item parametr\_1 - pozice kamene
			\item parametr\_x - index (pozice) lokace tahu
		\end{itemize}
\end{itemize}

\subsubsection{Kódy a zprávy odpovědí}
\begin{itemize}
	\item Pozitivní
		\begin{itemize}
			\item 202 - Turn Successful
		\end{itemize}
	\item Negativní
		\begin{itemize}
			\item 402 -
			\item 402 -
		\end{itemize}
\end{itemize}
%
%
%
\subsection{DISCONNECT}
\subsubsection{Popis}
Použito když se hráč odpojuje od serveru. Vymaže hráče ze seznamu hráčů.

\subsubsection{Přesný formáty}
\begin{itemize}
	\item ID\_hráče$|$DISCONNECT
\end{itemize}

\subsubsection{Kódy a zprávy odpovědí}
\begin{itemize}
	\item Pozitivní
		\begin{itemize}
			\item 201 - You were disconnected
		\end{itemize}
	\item Negativní
		\begin{itemize}
			\item 402 -
			\item 402 -
		\end{itemize}
\end{itemize}
%
%
%
\subsection{OPPONENT\_JOIN}
\subsubsection{Popis}
Instrukce, kterou posílá server klientovi (protivníkovi hráče, který poslal instrukci JOIN\_GAME),
informující ho o tahu protivníka. Instrukce musí být potvrzena pomocí "OK" nebo "ERROR" odpovědi od klienta.

\subsubsection{Přesný formáty}
\begin{itemize}
	\item ID\_protivníka$|$OPPONENT\_JOIN$|$kód$|$zpárva$|$username
		\begin{itemize}
			\item kód - jedná se o zprávu serveru, takže klient očekává kód operace
			\item zpráva - podobně jako kód je očekávána klientem
			\item username - jméno hráče volající JOIN\_GAME
		\end{itemize}
\end{itemize}

\subsubsection{Kódy a zprávy odpovědí}
\begin{itemize}
	\item Pozitivní
		\begin{itemize}
			\item 201 - Successfully joined game
		\end{itemize}
	\item Negativní
		\begin{itemize}
			\item 402 -
			\item 402 -
		\end{itemize}
	\item Parametr - odpověď má za parametr jméno protivníka
\end{itemize}
%
%
%
\subsection{OPPONENT\_TURN}
\subsubsection{Popis}
Pošle klientovi tah protivníka. Potvrzována pomocí "OK" nebo "ERROR".

\subsubsection{Přesný formáty}
\begin{itemize}
	\item ID\_hráče$|$OPPONENT\_TURN$|$parametr\_1$|$parametr\_2$|$...
\end{itemize}

\subsubsection{Kódy a zprávy odpovědí}
\begin{itemize}
	\item Pozitivní
		\begin{itemize}
			\item 201 - Opponent moved
		\end{itemize}
	\item Negativní
		\begin{itemize}
			\item 402 -
			\item 402 -
		\end{itemize}
\end{itemize}

%
\chapter{Programátorská dokumentace}
%
\chapter{Uživatelská dokumentace}
\chapter{Závěr}

\end{document}
